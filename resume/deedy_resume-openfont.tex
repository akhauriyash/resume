\documentclass[]{deedy-resume-openfont}
\begin{document}

%%%%%%%%%%%%%%%%%%%%%%%%%%%%%%%%%%%%%%
%     LAST UPDATED DATE
%%%%%%%%%%%%%%%%%%%%%%%%%%%%%%%%%%%%%%
% \lastupdated
%%%%%%%%%%%%%%%%%%%%%%%%%%%%%%%%%%%%%%
%     TITLE NAME
%%%%%%%%%%%%%%%%%%%%%%%%%%%%%%%%%%%%%%
\namesection{Yash}{Akhauri\hspace{5mm}\includegraphics[width=1.5cm, height=1.5cm]{resumeQR.png}}{
\href{https://quirkyai.wordpress.com}{Blog  |  } 
\href{https://github.com/akhauriyash}{GitHub  |  } 
\href{https://www.linkedin.com/in/akhauriyash/}{LinkedIn  |  } 
\href{mailto:akhauri.yash@gmail.com}{akhauri.yash@gmail.com} | +91 78915 12802 
}
% \vspace{2mm}
% %%%%%%%%%%%%%%%%%%%%%%%%%%%%%%%%%%%%%%
% %     COLUMN ONE
% %%%%%%%%%%%%%%%%%%%%%%%%%%%%%%%%%%%%%%
% \begin{minipage}[t]{0.30\textwidth} 
% %%%%%%%%%%%%%%%%%%%%%%%%%%%%%%%%%%%%%% 
% %     EDUCATION
% %%%%%%%%%%%%%%%%%%%%%%%%%%%%%%%%%%%%%%
% \subsection{Education} 
% \vspace{2mm}
% \descript{BITS Pilani}
% {B.E. Electronics and Instrumentation}
% \newline
% \location{Expected May 2020}
% \sectionsep
% \descript{Ryan Global School}
% {IGCSE 2014 | Kharghar, India}
% \location{Percentage 90.1\%}
% \sectionsep
% \vspace{3mm}
% %%%%%%%%%%%%%%%%%%%%%%%%%%%%%%%%%%%%%%
% %     SKILLS
% %%%%%%%%%%%%%%%%%%%%%%%%%%%%%%%%%%%%%%
% \subsection{Skills} 
% \vspace{2mm}
% \location{Programming languages:}
% \textbullet{} Python \\
% \textbullet{}   C++ \\
% \textbullet{} Java \\
% \textbullet{} Mathematica \\
% \vspace{3mm}
% \location{Others:}
% \textbullet{} CUDA      \hfill    \textbullet{} OpenMP \\ 
% \textbullet{} libGDX     \hfill     \textbullet{} Android studio \\
% \textbullet{} Linux      \hfill     \textbullet{} Excel \\
% \textbullet{} Tensorflow   \hfill   \textbullet{} PyTorch  \\
% \sectionsep
% %%%%%%%%%%%%%%%%%%%%%%%%%%%%%%%%%%%%%%
% %     COLUMN TWO 
% %%%%%%%%%%%%%%%%%%%%%%%%%%%%%%%%%%%%%%
% \end{minipage} 
% \hfill
% \begin{minipage}[t]{0.68\textwidth} 
%%%%%%%%%%%%%%%%%%%%%%%%%%%%%%%%%%%%%%
%     EXPERIENCE
%%%%%%%%%%%%%%%%%%%%%%%%%%%%%%%%%%%%%%
%\intro{An undergraduate Junior  with a love for developing and a passion for computer vision. 
%        Enjoy scaling steep learning curves and constantly looking for multidisciplinary approaches to solve problems. %Currently on the lookout for opportunities in the field of Artificial Intelligence.}

\section{Education}
% \noindent\rule{24cm}{0.4pt}\\
% \sectionsep
\runsubsection{BITS Pilani}
\descript{| B.E. in Electronics and Instrumentation}
\location{Aug 2016 - May 2020 | RJ, India\\
\custombold{Fluent: Python3, PyTorch, Mathematica \\  
 Familiar: C++, Java, CUDA, OpenMP, Tensorflow, Android Studio, LibGDX, Docker}}
\sectionsep
\section{Experience}
% \vspace{2mm}
% \sectionsep
\runsubsection{Intel}
\descript{| Research Scientist}
\location{Jun 2020 | Bengaluru, IN}
\begin{tightemize}
\item Research Scientist at the Cloud Systems Research (CSR) Lab in the Systems and Software Research (SSR) Group.
\end{tightemize}
\sectionsep

\runsubsection{Xilinx Research}
\descript{| Visiting Researcher}
\location{Aug 2019 - May 2020 | Dublin, Ireland}
% \vspace{3.5mm}
\begin{tightemize}
\item Developed a library for co-design of neural network topologies and reconfigurable hardware that maps to an efficient FPGA implementation without the need for a custom accelerator architecture or a scheduler.
\item Targeted the Jet Substructure Classification task as part of CERN LMS L1 trigger experiments, used the library to deploy models with $10\times$ lower latency than FPGA4HEP designs.Demonstrated quantization library Brevitas to CERN \href{https://github.com/akhauriyash/FPGA4HEP-Brevitas/}{\texttt{[GitHub]}}.
\end{tightemize}

% \sectionsep
% \runsubsection{Wolfram}
% \descript{| Mentor}
% \location{June 2019 - July 2019    |    Waltham, Massachusetts}
% \begin{tightemize}
% \item Acting as a mentor and teaching assistant for the Wolfram Summer School and Wolfram High-School camps, specializing in Artificial Intelligence. 
% \end{tightemize}

\sectionsep
\runsubsection{Uraniom}
\descript{| Research Intern}
\location{Jan 2019 - Jul 2019 | France (Remote) }
\begin{tightemize}
\item Implementing semantic segmentation models for face transfer across GIFs, progressive GANs for realistic UV map generation and exploring effective weight-sharing strategies for neural networks under a research collaboration.
\end{tightemize}

\sectionsep
\runsubsection{Wolfram}
\descript{| Undergraduate Researcher}
\location{June 2018 - July 2018    |    Waltham, Massachusetts}
\begin{tightemize}
\item Developed HadaNet MLPs in the Wolfram Language and worked on C OpenMP kernels for GEMM and Convolutions using the Hadamard Binarization algorithm.  \href{https://github.com/akhauriyash/XNOR-Intel-ISA}{\texttt{[OpenMP kernel]}} | \href{https://github.com/akhauriyash/XNOR-convolution}{\texttt{[CUDA kernel]}} | \href{https://docs.google.com/document/d/18uynX2yDSWm1BVCtG3Rd4CRb6xHiRxbvprUBTb4lvjY/edit?usp=sharing}{\texttt{[Whitepaper]}}
\end{tightemize}

\sectionsep
% \end{minipage} 

%%%%%%%%%%%%%%%%%%%%%%%%%%%%%%%%%%%%%%
%     Talks/Publications
%%%%%%%%%%%%%%%%%%%%%%%%%%%%%%%%%%%%%%
% \vspace{5mm}
\section{Papers \& Grants}
% \vspace{2mm}
\descript{LogicNet: Constructing Neural Networks from Truth Tables \hfill \texttt{Arkansas | May 2020}}
Paper accepted at \custombold{FCCM'20} as a poster presentation.\\
% Submitting paper at \custombold{NeurIPS'19 ML for Systems Workshop}. \\ 
\sectionsep 
\vspace{1mm}
\descript{HadaNets: Flexible Quantization Strategies for Neural Networks\href{https://researchgate.net/publication/329950131_HadaNets_Flexible_Quantization_Strategies_for_Neural_Networks}{\texttt{[Link]}} \hfill \texttt{California | Jun 2019}}
Paper accepted at \custombold{CVPR'19 UAVision workshop - Orals}. \\ 
Delivered a "Theatre Talk" and poster at the Intel Demo Booth at CVPR'19.\\
\sectionsep
\vspace{1mm}
\descript{Wolfram Technology Conference -- Speaker \hfill \texttt{Champaign, IL | Oct. 2018}}
Delivered a talk on my research on Hadamard Neural Networks. \\
\sectionsep
\vspace{1mm}
\descript{Intel Nervana Early Innovators Grant \hfill \texttt{\$5000}}
Received research grant to develop Binary Precision Neural Networks and Real time Artistic Style Transfer. The technical article can be found  \href{https://software.intel.com/en-us/articles/art-em-artistic-style-transfer-to-virtual-reality-final-update}{\texttt{[here.]}}
The code can be found \href{https://github.com/akhauriyash/Fast-style-transfer}{\texttt{[here.]}}\\
\sectionsep 
\vspace{1mm}
\descript{Intel CVPR Travel Grant \hfill \texttt{\$3000}}
% Received a travel grant from Intel to present at the Intel Demo Booth at CVPR'19.\\
\sectionsep
\vspace{1mm}
\descript{Wolfram Student Aid \hfill \texttt{\$2400}}
Received aid to attend the Wolfram Summer School and develop Hadamard Binary Neural Networks. \\
\sectionsep
\vspace{1mm}
\descript{Intel AI Meetup -- Speaker \href{https://docs.google.com/presentation/d/1PuxUM6RmDI5Dq7XgFoVWNxv_TOQ3aux8GiQ9aWlpUjg/edit?usp=sharing}{\texttt{[pptx]}} \href{https://software.intel.com/en-us/articles/developing-hadamard-neural-networks-on-the-intel-xeon-scalable-processor}{\texttt{[Article]}} \hfill \texttt{Delhi, IN | Sept. 2018}}
Spoke about my research on scaling AI using Intel technologies. This event was organized by Intel. \\
% \sectionsep
% \vspace{1mm}
% \descript{Intel AI Academy Success Story \href{https://software.intel.com/en-us/articles/developing-hadamard-neural-networks-on-the-intel-xeon-scalable-processor}{\texttt{[Link]}}}
% Intel published a cover story of my research done in the field of Quantized Neural Networks.\\
\sectionsep
\vspace{1mm}
\descript{Intel AI DevCon -- Poster \hfill \texttt{San Francisco, Bangalore | May \& Aug 2018}}
% Presented posters on quantized GEMM kernels for Intel Xeon Phi\\
% \sectionsep
% \vspace{1mm}
% \descript{KVPY Scholar}
% Selected as a KVPY scholar by the Department of Science and Technology, Government of India. \\
% \sectionsep
% \descript{INSPIRE Scholarship}
% Selected for the INSPIRE Scholarship by the Department of Science and Technology (DST), Government of India.
\clearpage
% %%%%%%%%%%%%%%%%%%%%%%%%%%%%%%%%%%%%%
% %     PROJECTS      
% %%%%%%%%%%%%%%%%%%%%%%%%%%%%%%%%%%%%%%
% \vspace{2mm} 
% \section{  }
% \section{Projects}
% \vspace{2mm}
% \descript{EncoderNets: Maximizing encoders, Minimizing Cache-Misses. \hfill \texttt{PyTorch}}
% Developing post-training quantization strategies to build networks for embedded devices.\\
% \sectionsep
% \vspace{1mm}
% \descript{Whitepaper - Improving distributed mesh computing with Hadamard Binary Neural Networks.}
% \href{https://docs.google.com/document/d/18uynX2yDSWm1BVCtG3Rd4CRb6xHiRxbvprUBTb4lvjY/edit?usp=sharing}{\texttt{[Link]}}\\
% \sectionsep
% \vspace{1mm}
% \descript{xGEMM \& xCONV \hfill \texttt{C++, CUDA, OpenMP}}
% Coded efficient 3D convolutional and GEMM kernels for XNOR (bit quantized) networks using CUDA C programming and OpenMP. Optimized kernels are for Intel processors and Nvidia GPUs. Invited to present a poster at Intel AI DevCon and Intel AI Student Ambassador Summit, San Francisco. The codes can be found here: \href{https://github.com/akhauriyash/XNOR-Intel-ISA}{\texttt{[OpenMP kernel]}} | \href{https://github.com/akhauriyash/XNOR-convolution}{\texttt{[CUDA kernel]}}.\\
% \sectionsep
% \vspace{1mm}
% \descript{GEMM: From Pure C to SSE Optimized Micro Kernels \hfill \texttt{C++}}
% Followed 
%  \href{http://apfel.mathematik.uni-ulm.de/~lehn/sghpc/gemm/}{\texttt{this tutorial}} to build an implementation of GEMM that achieves performance close to that of the BLIS kernels. \\
% \sectionsep
% \vspace{1mm}
% \descript{Real time artistic style transfer \hfill \texttt{Python, Tensorflow, OpenCV}}/
% \sectionsep
% \vspace{1mm}
% \descript{GravDash \hfill \texttt{Java, LibGDX, Android Studio}}
% Developed an android game using Java and libGDX as the framework. The game can be found  \href{https://play.google.com/store/apps/details?id=com.mygdx.gravtry2&hl=en}{\texttt{[here.]}}\\
% \sectionsep
% \vspace{1mm}
% \descript{Blog \hfill  \texttt{Python, Java, C++, Tensorflow, OpenMP}}
% Maintaining a \href{https://quirkyai.wordpress.com}{\texttt{[blog]}} covering various AI   related topics with over 10000 hits. \\
% \sectionsep


% \sectionsep
\end{document}
